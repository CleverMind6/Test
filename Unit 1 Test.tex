\documentclass[11pt]{article}
\usepackage{geometry, graphicx, amsmath, amsthm, amssymb, afterpage}
\usepackage[labelsep=period, labelfont=bf]{caption}
\geometry{margin=0.75in}

\begin{document}
{\LARGE Unit 1 Practice Test}

\hrulefill

\begin{enumerate}
% Topics: Absolute Value Intervals, Transformations, Piecewise Functions, Domain and Range, Even vs Odd Functions
\item If an interval is given, draw the interval on the number line. If a number line range is given, determine the interval in terms of absolute values.
\begin{enumerate}
\item \(|x| > 1\)
\vspace{2cm}

\item \hrulefill
\begin{figure}[h]\centering
\includegraphics[scale=0.5]{graph1}
\end{figure}

\item \(|x|<3\)
\vspace{2cm}

\item \(|x| \leq 7\)
\vspace{2cm}

\item \hrulefill
\begin{figure}[h]\centering
\includegraphics[scale=0.5]{graph2}
\end{figure}

\item \hrulefill
\begin{figure}[h]\centering
\includegraphics[scale=0.5]{graph3}
\end{figure}

\item \(|x| \geq 4\)
\vspace{2cm}

\end{enumerate}

\item Determine if each of the following functions are even, odd, both, or neither.
\begin{enumerate}
\item \(f(x) = x^2 - 3\) \hrulefill
\item \(g(x) = x^3\) \hrulefill
\item \(h(x) = 0\) \hrulefill
\item \(j(x) = 12\) \hrulefill
\item \(l(x) = \tfrac{13}{x}\) \hrulefill
\end{enumerate}
\pagebreak
\item Prove that \(f(x) = 2^x + 2^{-x}\) is even.
\vspace{3in}

\item For each of the following functions, 
\renewcommand{\labelenumii}{\roman{enumii})}
\begin{enumerate}
\setcounter{enumii}{0}
\item State the domain and range.
\item State the base function.
\item State the transformations (vertical stretch \textit{a}, vertical translation \textit{c}, horizontal compression \textit{k}, horizontal translation \textit{d}, and any reflections). 
\item Draw the function with a smooth line and its base function with a dotted line.
\item Determine if the inverse relation of the function is the last page or not. If so, state the figure number.
\end{enumerate}
\renewcommand{\labelenumii}{(\alph{enumii})}
\begin{enumerate}
\item \(y(x) = 8x - 4\)
\vspace{4.5in}
\item \(f(x) = |2(x-3)|+1\)
\vspace{4.5in}
\item \(g(x) = -\dfrac{1}{2}(x+2)^3-3 \)
\vspace{4.5in}
\item \(h(x) = \dfrac{3}{x+1} \)
\vspace{4.5in}
\item \(A(x) = \pi x^2\)
\vspace{4.5in}
\end{enumerate}
\pagebreak

\item Consider the following piecewise function.
\[f(x) = \left\{
\begin{array}{cc}
2(x+5)^2	& x < -4	\\
\left|\dfrac{x}{2} \right|	& -4 \leq x \leq 1	\\
x	& 2 < x
\end{array}
\right. \]
\begin{enumerate}
\item State the domain, range, and transformations for each of the pieces of the piecewise function \(f(x)\). Keep in mind the domain is controlled by the piecewise function.
\begin{enumerate}
\item When \(x<-4\):
\vspace{3in}

\item When \(-4\leq x \leq 1\):
\vspace{3in}

\item When \(2< x\):
\vspace{3in}
\end{enumerate}

\item Evaluate the following:
\begin{enumerate}
\item \(g(-4)\) where \(g(x)=2(x+5)^2\).
\vspace{2in}

\item \(h(-4)\) and \(h(1)\) where \(h(x) = \left|\dfrac{x}{2}\right| \).
\vspace{4in}
\end{enumerate}
\item Is the function discontinuous anywhere? If so, state where.
\pagebreak
\item Graph the function. 
\vspace{4.5in}
\item Graph its inverse relation.
\pagebreak
\end{enumerate}
% \ffigbox[<width>][<height>][<vert pos>]{<caption>}{<object>}
% \FBwidth = natural object width.

\begin{figure}
\centering
\begin{minipage}{0.45\textwidth}
\centering
\includegraphics[scale=0.23]{f1}
\caption{}
\end{minipage}\hfill
\begin{minipage}{0.45\textwidth}
\centering
\includegraphics[scale=0.23]{f2}
\caption{}
\end{minipage}
\end{figure}

\begin{figure}
\centering
\begin{minipage}{0.45\textwidth}
\centering
\includegraphics[scale=0.23]{f3}
\caption{}
\end{minipage}\hfill
\begin{minipage}{0.45\textwidth}
\centering
\includegraphics[scale=0.23]{f4}
\caption{}
\end{minipage}
\end{figure}

\begin{figure}
\centering
\begin{minipage}{0.45\textwidth}
\centering
\includegraphics[scale=0.23]{f5}
\caption{}
\end{minipage}\hfill
\begin{minipage}{0.45\textwidth}
\centering
\includegraphics[scale=0.23]{f6}
\caption{}
\end{minipage}
\end{figure}

\begin{figure}
\centering
\begin{minipage}{0.45\textwidth}
\centering
\includegraphics[scale=0.23]{f7}
\caption{}
\end{minipage}\hfill
\begin{minipage}{0.45\textwidth}
\centering
\includegraphics[scale=0.23]{f8}
\caption{}
\end{minipage}
\end{figure}

\end{enumerate}
\end{document}